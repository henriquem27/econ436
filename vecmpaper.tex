% Options for packages loaded elsewhere
\PassOptionsToPackage{unicode}{hyperref}
\PassOptionsToPackage{hyphens}{url}
%
\documentclass[
]{article}
\title{VECMPAPER}
\author{}
\date{\vspace{-2.5em}}

\usepackage{amsmath,amssymb}
\usepackage{lmodern}
\usepackage{iftex}
\ifPDFTeX
  \usepackage[T1]{fontenc}
  \usepackage[utf8]{inputenc}
  \usepackage{textcomp} % provide euro and other symbols
\else % if luatex or xetex
  \usepackage{unicode-math}
  \defaultfontfeatures{Scale=MatchLowercase}
  \defaultfontfeatures[\rmfamily]{Ligatures=TeX,Scale=1}
\fi
% Use upquote if available, for straight quotes in verbatim environments
\IfFileExists{upquote.sty}{\usepackage{upquote}}{}
\IfFileExists{microtype.sty}{% use microtype if available
  \usepackage[]{microtype}
  \UseMicrotypeSet[protrusion]{basicmath} % disable protrusion for tt fonts
}{}
\makeatletter
\@ifundefined{KOMAClassName}{% if non-KOMA class
  \IfFileExists{parskip.sty}{%
    \usepackage{parskip}
  }{% else
    \setlength{\parindent}{0pt}
    \setlength{\parskip}{6pt plus 2pt minus 1pt}}
}{% if KOMA class
  \KOMAoptions{parskip=half}}
\makeatother
\usepackage{xcolor}
\IfFileExists{xurl.sty}{\usepackage{xurl}}{} % add URL line breaks if available
\IfFileExists{bookmark.sty}{\usepackage{bookmark}}{\usepackage{hyperref}}
\hypersetup{
  pdftitle={VECMPAPER},
  hidelinks,
  pdfcreator={LaTeX via pandoc}}
\urlstyle{same} % disable monospaced font for URLs
\usepackage[margin=1in]{geometry}
\usepackage{color}
\usepackage{fancyvrb}
\newcommand{\VerbBar}{|}
\newcommand{\VERB}{\Verb[commandchars=\\\{\}]}
\DefineVerbatimEnvironment{Highlighting}{Verbatim}{commandchars=\\\{\}}
% Add ',fontsize=\small' for more characters per line
\usepackage{framed}
\definecolor{shadecolor}{RGB}{248,248,248}
\newenvironment{Shaded}{\begin{snugshade}}{\end{snugshade}}
\newcommand{\AlertTok}[1]{\textcolor[rgb]{0.94,0.16,0.16}{#1}}
\newcommand{\AnnotationTok}[1]{\textcolor[rgb]{0.56,0.35,0.01}{\textbf{\textit{#1}}}}
\newcommand{\AttributeTok}[1]{\textcolor[rgb]{0.77,0.63,0.00}{#1}}
\newcommand{\BaseNTok}[1]{\textcolor[rgb]{0.00,0.00,0.81}{#1}}
\newcommand{\BuiltInTok}[1]{#1}
\newcommand{\CharTok}[1]{\textcolor[rgb]{0.31,0.60,0.02}{#1}}
\newcommand{\CommentTok}[1]{\textcolor[rgb]{0.56,0.35,0.01}{\textit{#1}}}
\newcommand{\CommentVarTok}[1]{\textcolor[rgb]{0.56,0.35,0.01}{\textbf{\textit{#1}}}}
\newcommand{\ConstantTok}[1]{\textcolor[rgb]{0.00,0.00,0.00}{#1}}
\newcommand{\ControlFlowTok}[1]{\textcolor[rgb]{0.13,0.29,0.53}{\textbf{#1}}}
\newcommand{\DataTypeTok}[1]{\textcolor[rgb]{0.13,0.29,0.53}{#1}}
\newcommand{\DecValTok}[1]{\textcolor[rgb]{0.00,0.00,0.81}{#1}}
\newcommand{\DocumentationTok}[1]{\textcolor[rgb]{0.56,0.35,0.01}{\textbf{\textit{#1}}}}
\newcommand{\ErrorTok}[1]{\textcolor[rgb]{0.64,0.00,0.00}{\textbf{#1}}}
\newcommand{\ExtensionTok}[1]{#1}
\newcommand{\FloatTok}[1]{\textcolor[rgb]{0.00,0.00,0.81}{#1}}
\newcommand{\FunctionTok}[1]{\textcolor[rgb]{0.00,0.00,0.00}{#1}}
\newcommand{\ImportTok}[1]{#1}
\newcommand{\InformationTok}[1]{\textcolor[rgb]{0.56,0.35,0.01}{\textbf{\textit{#1}}}}
\newcommand{\KeywordTok}[1]{\textcolor[rgb]{0.13,0.29,0.53}{\textbf{#1}}}
\newcommand{\NormalTok}[1]{#1}
\newcommand{\OperatorTok}[1]{\textcolor[rgb]{0.81,0.36,0.00}{\textbf{#1}}}
\newcommand{\OtherTok}[1]{\textcolor[rgb]{0.56,0.35,0.01}{#1}}
\newcommand{\PreprocessorTok}[1]{\textcolor[rgb]{0.56,0.35,0.01}{\textit{#1}}}
\newcommand{\RegionMarkerTok}[1]{#1}
\newcommand{\SpecialCharTok}[1]{\textcolor[rgb]{0.00,0.00,0.00}{#1}}
\newcommand{\SpecialStringTok}[1]{\textcolor[rgb]{0.31,0.60,0.02}{#1}}
\newcommand{\StringTok}[1]{\textcolor[rgb]{0.31,0.60,0.02}{#1}}
\newcommand{\VariableTok}[1]{\textcolor[rgb]{0.00,0.00,0.00}{#1}}
\newcommand{\VerbatimStringTok}[1]{\textcolor[rgb]{0.31,0.60,0.02}{#1}}
\newcommand{\WarningTok}[1]{\textcolor[rgb]{0.56,0.35,0.01}{\textbf{\textit{#1}}}}
\usepackage{longtable,booktabs,array}
\usepackage{calc} % for calculating minipage widths
% Correct order of tables after \paragraph or \subparagraph
\usepackage{etoolbox}
\makeatletter
\patchcmd\longtable{\par}{\if@noskipsec\mbox{}\fi\par}{}{}
\makeatother
% Allow footnotes in longtable head/foot
\IfFileExists{footnotehyper.sty}{\usepackage{footnotehyper}}{\usepackage{footnote}}
\makesavenoteenv{longtable}
\usepackage{graphicx}
\makeatletter
\def\maxwidth{\ifdim\Gin@nat@width>\linewidth\linewidth\else\Gin@nat@width\fi}
\def\maxheight{\ifdim\Gin@nat@height>\textheight\textheight\else\Gin@nat@height\fi}
\makeatother
% Scale images if necessary, so that they will not overflow the page
% margins by default, and it is still possible to overwrite the defaults
% using explicit options in \includegraphics[width, height, ...]{}
\setkeys{Gin}{width=\maxwidth,height=\maxheight,keepaspectratio}
% Set default figure placement to htbp
\makeatletter
\def\fps@figure{htbp}
\makeatother
\setlength{\emergencystretch}{3em} % prevent overfull lines
\providecommand{\tightlist}{%
  \setlength{\itemsep}{0pt}\setlength{\parskip}{0pt}}
\setcounter{secnumdepth}{-\maxdimen} % remove section numbering
\ifLuaTeX
  \usepackage{selnolig}  % disable illegal ligatures
\fi

\begin{document}
\maketitle

\hypertarget{introduction}{%
\section{Introduction}\label{introduction}}

In this paper I will analyze Crypto currencies, and for that I choose 6
different coins which are Bitcoin, Ethereum, Cardano, and Waves, this
choice of coins was mostly due to the data available as the Crypto
market is new and so are some of the coins, also, I made sure to have
balance between a few more established coins such as Bitcoin and
Ethereum while the others are still new and have not yet fully been
established.I got the data from Yahoo Finance, and it includes a range
from 221 to 385 weekly observations depending on the coin, for my
project I intend to use closing price of the week, however, the data
also included opening, high and low price of the week and volume of
trades for the week. Daily data was also available, however,
cryptocurrencies are extremely volatile which caused daily data to be
quite noisy, which is why I choose weekly data despite lowering the
number of observations.For the periods, I selected to remove the last 10
weeks and estimate the models on the rest of the data which is from
2017-11-06 to 2022-01-10.

\hypertarget{vector-error-correction-model}{%
\section{Vector Error Correction
Model}\label{vector-error-correction-model}}

Bitcoin is the oldest and most dominant cryptocurrency its price is
almost 3 times the price of its closest competitor Ethereum, for many it
is considered as ``digital gold'' and a lot of the cheaper currencies
can only be purchased with Bitcoin, therefore, its very likely that the
movements in Bitcoin price are likely to be very impactful throughout
the cryptocurrency market as a lot of the thrust in the cryptocurrency
market has been due to Bitcoin's success, however, it is quite unlikely
that movements on Cardano which costs a fraction of the Price of Bitcoin
have any impact whatsoever on its price.

Due to the characteristics of the cryptocurrency market discussed above,
I decided to estimate the VECM of lag 25 with 3 ranks where Cardano,
Ethereum and Waves all depend only on Bitcoin, from that we get the
following equations:

\(Equation 1 : Log(Cardano)_t-{2.0178}*Log(Bitcoin)_t=e_t\) (0.20963)
\(Equation 2 : Log(Ethereum)_t-{1.6277}*Log(Bitcoin)_t=u_t\) (0.15864)
\(Equation 3 : Log(Waves)_t-{1.8296}*Log(Bitcoin)_t=v_t\) (0.29532)

As we can see by the-negative Bitcoin coefficients in all equations, we
have positive relationships, for example if Bitcoin price increases
Cardano prices also have to increase in order for the equation return to
equilibrium. This relationship was what was expected since as previously
mentioned Bitcoin is the driving force of the crypto market and
therefore, if its price goes up it is highly likely that the other
cryptocurrencies will follow and if it's price goes down it is also
likely that the other cryptocurrencies prices will go down, which is
also what the VECM model predicted. In terms of the speeds of adjustment
only Cardano clears the vectors with an EC1 estimate of −0.362121 and
significant p-value of less then 0.005, which means that Cardano will
move to clear the vector when there is variation between the 2 series.

\hypertarget{var-lag-selection}{%
\section{VAR Lag Selection}\label{var-lag-selection}}

For the VAR lags I decided to use 15, as from the previous paper and
Testing on the new period it seems to be the lag with the lowest root
mean squared error. I also tried to estimate the using the same lag as
the VECM ,however, it did not lower the RMSE therefore, I have decided
to stay with 15 lags for the VAR.

\hypertarget{bitcoin}{%
\section{Bitcoin}\label{bitcoin}}

\hypertarget{arma015}{%
\subsection{ARMA(0,15)}\label{arma015}}

The MA 15 process was choosen based on the first paper and the lowest
RMSE for this period, while also making the error are white noise in the
ACF.

\begin{verbatim}
## [1] "ARMA RMSE: 0.110146446508882"
\end{verbatim}

\hypertarget{var}{%
\subsection{VAR}\label{var}}

\begin{verbatim}
## [1] "VAR RMSE: 0.0811068177570365"
\end{verbatim}

\hypertarget{vecm}{%
\subsection{VECM}\label{vecm}}

\begin{Shaded}
\begin{Highlighting}[]
\NormalTok{mse }\OtherTok{\textless{}{-}} \FunctionTok{mean}\NormalTok{(vecm}\SpecialCharTok{$}\NormalTok{residuals[,}\DecValTok{4}\NormalTok{]}\SpecialCharTok{\^{}}\DecValTok{2}\NormalTok{)}

\NormalTok{rmvecm }\OtherTok{\textless{}{-}} \FunctionTok{sqrt}\NormalTok{(mse)}

\FunctionTok{paste}\NormalTok{(}\StringTok{"VECM RMSE:"}\NormalTok{,rmvecm)}
\end{Highlighting}
\end{Shaded}

\begin{verbatim}
## [1] "VECM RMSE: 0.0681983325531644"
\end{verbatim}

\begin{Shaded}
\begin{Highlighting}[]
\NormalTok{pred}\OtherTok{\textless{}{-}}\FunctionTok{predict}\NormalTok{(vecm,}\AttributeTok{n.ahead =} \DecValTok{10}\NormalTok{)}

\FunctionTok{ggplot}\NormalTok{()}\SpecialCharTok{+}\FunctionTok{geom\_line}\NormalTok{(}\FunctionTok{aes}\NormalTok{(}\AttributeTok{x=}\FunctionTok{as.Date}\NormalTok{(data}\SpecialCharTok{$}\NormalTok{Date),}\AttributeTok{y=}\FunctionTok{log}\NormalTok{(data}\SpecialCharTok{$}\NormalTok{Bitcoin)),}\AttributeTok{group=}\DecValTok{1}\NormalTok{)}\SpecialCharTok{+}\FunctionTok{geom\_line}\NormalTok{(}\FunctionTok{aes}\NormalTok{(}\AttributeTok{x=}\FunctionTok{as.Date}\NormalTok{(data}\SpecialCharTok{$}\NormalTok{Date[}\DecValTok{220}\SpecialCharTok{:}\DecValTok{229}\NormalTok{]),}\AttributeTok{y=}\NormalTok{pred[,}\DecValTok{4}\NormalTok{]),}\AttributeTok{group=}\DecValTok{1}\NormalTok{,}\AttributeTok{color=}\StringTok{"red"}\NormalTok{)}\SpecialCharTok{+}\FunctionTok{labs}\NormalTok{(}\AttributeTok{title=}\StringTok{"VECM Forecast For Bitcoin"}\NormalTok{,}\AttributeTok{x=}\StringTok{"Date"}\NormalTok{,}\AttributeTok{y=}\StringTok{"Log of Price"}\NormalTok{)}
\end{Highlighting}
\end{Shaded}

\includegraphics{vecmpaper_files/figure-latex/unnamed-chunk-4-1.pdf}

For Bitcoin we can see that the VECM does a lot better in terms of the
RMSE being 0.02 lower than the VAR and 0.05 lower the the Univariate
approach with MA15, the forecast seem to capture the movement of
somewhat well as despite getting the first week and last week wrong it
does quite well on the other weeks successfully capturing the up and
down trend of the series.

\hypertarget{ethereum}{%
\section{Ethereum}\label{ethereum}}

\hypertarget{arma017}{%
\subsection{ARMA(0,17)}\label{arma017}}

The MA 17 process was choosen based on the first paper and the lowest
RMSE for this period, while also making the error are white noise in the
ACF.

\begin{verbatim}
## [1] "ARMA RMSE: 0.14230004508759"
\end{verbatim}

\hypertarget{var-1}{%
\subsection{VAR}\label{var-1}}

\begin{verbatim}
## [1] "VAR RMSE: 0.108871417792549"
\end{verbatim}

\hypertarget{vecm-1}{%
\subsection{VECM}\label{vecm-1}}

\begin{Shaded}
\begin{Highlighting}[]
\NormalTok{mse }\OtherTok{\textless{}{-}} \FunctionTok{mean}\NormalTok{(vecm}\SpecialCharTok{$}\NormalTok{residuals[,}\DecValTok{3}\NormalTok{]}\SpecialCharTok{\^{}}\DecValTok{2}\NormalTok{)}

\NormalTok{rmvecm }\OtherTok{\textless{}{-}} \FunctionTok{sqrt}\NormalTok{(mse)}

\FunctionTok{paste}\NormalTok{(}\StringTok{"VECM RMSE:"}\NormalTok{,rmvecm)}
\end{Highlighting}
\end{Shaded}

\begin{verbatim}
## [1] "VECM RMSE: 0.0856005276682794"
\end{verbatim}

\begin{Shaded}
\begin{Highlighting}[]
\NormalTok{pred}\OtherTok{\textless{}{-}}\FunctionTok{predict}\NormalTok{(vecm,}\AttributeTok{n.ahead =} \DecValTok{10}\NormalTok{)}

\FunctionTok{ggplot}\NormalTok{()}\SpecialCharTok{+}\FunctionTok{geom\_line}\NormalTok{(}\FunctionTok{aes}\NormalTok{(}\AttributeTok{x=}\FunctionTok{as.Date}\NormalTok{(data}\SpecialCharTok{$}\NormalTok{Date),}\AttributeTok{y=}\FunctionTok{log}\NormalTok{(data}\SpecialCharTok{$}\NormalTok{Ethereum)),}\AttributeTok{group=}\DecValTok{1}\NormalTok{)}\SpecialCharTok{+}\FunctionTok{geom\_line}\NormalTok{(}\FunctionTok{aes}\NormalTok{(}\AttributeTok{x=}\FunctionTok{as.Date}\NormalTok{(data}\SpecialCharTok{$}\NormalTok{Date[}\DecValTok{220}\SpecialCharTok{:}\DecValTok{229}\NormalTok{]),}\AttributeTok{y=}\NormalTok{pred[,}\DecValTok{3}\NormalTok{]),}\AttributeTok{group=}\DecValTok{1}\NormalTok{,}\AttributeTok{color=}\StringTok{"red"}\NormalTok{)}\SpecialCharTok{+}\FunctionTok{labs}\NormalTok{(}\AttributeTok{title=}\StringTok{"VECM Forecast For Ethereum"}\NormalTok{,}\AttributeTok{x=}\StringTok{"Date"}\NormalTok{,}\AttributeTok{y=}\StringTok{"Log of Price"}\NormalTok{)}
\end{Highlighting}
\end{Shaded}

\includegraphics{vecmpaper_files/figure-latex/unnamed-chunk-7-1.pdf}

In the case of Ethereum VECM is still much better in terms of the RMSE
when compared to the Univariate approach and the VAR, for the forecast
it seems that the VECM captures the trend very well up until the last
week where it predicts a downward trend whereas the actual data is a
upward trend.

\hypertarget{cardano}{%
\section{Cardano}\label{cardano}}

\hypertarget{arma09}{%
\subsection{ARMA(0,9)}\label{arma09}}

The MA 9 process was choosen based on the first paper and the lowest
RMSE for this period, while also making the error are white noise in the
ACF.

\begin{verbatim}
## [1] "ARMA RMSE: 0.202436044348384"
\end{verbatim}

\hypertarget{var-2}{%
\subsection{VAR}\label{var-2}}

\begin{verbatim}
## [1] "VAR RMSE: 0.127080566668958"
\end{verbatim}

\hypertarget{vecm-2}{%
\subsection{VECM}\label{vecm-2}}

\begin{Shaded}
\begin{Highlighting}[]
\NormalTok{mse }\OtherTok{\textless{}{-}} \FunctionTok{mean}\NormalTok{(vecm}\SpecialCharTok{$}\NormalTok{residuals[,}\DecValTok{2}\NormalTok{]}\SpecialCharTok{\^{}}\DecValTok{2}\NormalTok{)}

\NormalTok{rmvecm }\OtherTok{\textless{}{-}} \FunctionTok{sqrt}\NormalTok{(mse)}

\FunctionTok{paste}\NormalTok{(}\StringTok{"VECM RMSE:"}\NormalTok{,rmvecm)}
\end{Highlighting}
\end{Shaded}

\begin{verbatim}
## [1] "VECM RMSE: 0.0967846740423259"
\end{verbatim}

\begin{Shaded}
\begin{Highlighting}[]
\NormalTok{pred}\OtherTok{\textless{}{-}}\FunctionTok{predict}\NormalTok{(vecm,}\AttributeTok{n.ahead =} \DecValTok{10}\NormalTok{)}

\FunctionTok{ggplot}\NormalTok{()}\SpecialCharTok{+}\FunctionTok{geom\_line}\NormalTok{(}\FunctionTok{aes}\NormalTok{(}\AttributeTok{x=}\FunctionTok{as.Date}\NormalTok{(data}\SpecialCharTok{$}\NormalTok{Date),}\AttributeTok{y=}\FunctionTok{log}\NormalTok{(data}\SpecialCharTok{$}\NormalTok{Cardano)),}\AttributeTok{group=}\DecValTok{1}\NormalTok{)}\SpecialCharTok{+}\FunctionTok{geom\_line}\NormalTok{(}\FunctionTok{aes}\NormalTok{(}\AttributeTok{x=}\FunctionTok{as.Date}\NormalTok{(data}\SpecialCharTok{$}\NormalTok{Date[}\DecValTok{220}\SpecialCharTok{:}\DecValTok{229}\NormalTok{]),}\AttributeTok{y=}\NormalTok{pred[,}\DecValTok{2}\NormalTok{]),}\AttributeTok{group=}\DecValTok{1}\NormalTok{,}\AttributeTok{color=}\StringTok{"red"}\NormalTok{)}\SpecialCharTok{+}\FunctionTok{labs}\NormalTok{(}\AttributeTok{title=}\StringTok{"VECM Forecast For Cardano"}\NormalTok{,}\AttributeTok{x=}\StringTok{"Date"}\NormalTok{,}\AttributeTok{y=}\StringTok{"Log of Price"}\NormalTok{)}
\end{Highlighting}
\end{Shaded}

\includegraphics{vecmpaper_files/figure-latex/unnamed-chunk-10-1.pdf}

For Cardano we have quite a large difference, between the Univariate
model RMSE (0.20) and the other models VAR (0.12) and VECM (0.09) RMSE,
with VECM also performing the best for Cardano. For the VECM, the
forecast does not do very well as it misses most of the downward trend
in the first few weeks.

\hypertarget{waves}{%
\section{Waves}\label{waves}}

\hypertarget{arma017-1}{%
\subsection{ARMA(0,17)}\label{arma017-1}}

The MA 17 process was choosen based on the first paper and the lowest
RMSE for this period, while also making the error are white noise in the
ACF.

\begin{verbatim}
## [1] "ARMA RMSE: 0.171533674791841"
\end{verbatim}

\hypertarget{var-3}{%
\subsection{VAR}\label{var-3}}

\begin{verbatim}
## [1] "VAR RMSE: 0.131526318599454"
\end{verbatim}

\hypertarget{vecm-3}{%
\subsection{VECM}\label{vecm-3}}

\begin{Shaded}
\begin{Highlighting}[]
\NormalTok{mse }\OtherTok{\textless{}{-}} \FunctionTok{mean}\NormalTok{(vecm}\SpecialCharTok{$}\NormalTok{residuals[,}\DecValTok{1}\NormalTok{]}\SpecialCharTok{\^{}}\DecValTok{2}\NormalTok{)}

\NormalTok{rmvecm }\OtherTok{\textless{}{-}} \FunctionTok{sqrt}\NormalTok{(mse)}

\FunctionTok{paste}\NormalTok{(}\StringTok{"VECM RMSE:"}\NormalTok{,rmvecm)}
\end{Highlighting}
\end{Shaded}

\begin{verbatim}
## [1] "VECM RMSE: 0.104218786843637"
\end{verbatim}

\begin{Shaded}
\begin{Highlighting}[]
\NormalTok{pred}\OtherTok{\textless{}{-}}\FunctionTok{predict}\NormalTok{(vecm,}\AttributeTok{n.ahead =} \DecValTok{10}\NormalTok{)}

\FunctionTok{ggplot}\NormalTok{()}\SpecialCharTok{+}\FunctionTok{geom\_line}\NormalTok{(}\FunctionTok{aes}\NormalTok{(}\AttributeTok{x=}\FunctionTok{as.Date}\NormalTok{(data}\SpecialCharTok{$}\NormalTok{Date),}\AttributeTok{y=}\FunctionTok{log}\NormalTok{(data}\SpecialCharTok{$}\NormalTok{Waves)),}\AttributeTok{group=}\DecValTok{1}\NormalTok{)}\SpecialCharTok{+}\FunctionTok{geom\_line}\NormalTok{(}\FunctionTok{aes}\NormalTok{(}\AttributeTok{x=}\FunctionTok{as.Date}\NormalTok{(data}\SpecialCharTok{$}\NormalTok{Date[}\DecValTok{220}\SpecialCharTok{:}\DecValTok{229}\NormalTok{]),}\AttributeTok{y=}\NormalTok{pred[,}\DecValTok{1}\NormalTok{]),}\AttributeTok{group=}\DecValTok{1}\NormalTok{,}\AttributeTok{color=}\StringTok{"red"}\NormalTok{)}\SpecialCharTok{+}\FunctionTok{labs}\NormalTok{(}\AttributeTok{title=}\StringTok{"VECM Forecast For Waves"}\NormalTok{,}\AttributeTok{x=}\StringTok{"Date"}\NormalTok{,}\AttributeTok{y=}\StringTok{"Log of Price"}\NormalTok{)}
\end{Highlighting}
\end{Shaded}

\includegraphics{vecmpaper_files/figure-latex/unnamed-chunk-13-1.pdf}

Continuing the trend in terms of RMSE, VECM does the best for Waves with
the univariate approach being the worst, with VAR in the middle but
still better than the univariate. Despite the low RMSE, the VECM model
completely misses the upward trend in the forecast by quite a lot, as it
seems that waves is bouncing back from a big downturn which the vecm
struggles to capture.

\hypertarget{conclusion}{%
\section{Conclusion}\label{conclusion}}

\begin{longtable}[]{@{}rlll@{}}
\toprule
Variables & ARMA RMSE & VAR RMSE & VECM RMSE \\
\midrule
\endhead
BITCOIN & 0.110146446508882 & 0.08110681775703654 &
0.0681983325531644 \\
Ethereum & 0.14230004508759 & 0.108871417792549 & 0.0856005276682794 \\
Cardano & 0.202436044348384 & 0.127080566668958 & 0.0967846740423259 \\
Waves & 0.171533674791841 & 0.131526318599454 & 0.104218786843637 \\
\bottomrule
\end{longtable}

\end{document}
